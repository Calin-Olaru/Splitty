\section{Assignment Description}
In this project, we, as a team need to develop an application that helps people better organize their money spent when going out with friends. It helps people settle their debts more easily when splitting the costs between them.
\\\\
Our app must be able to connect multiple clients to a server simultaneously. It enables clients to create and participate in events where they can manage their expenses related to those events. It also allows them to create invite codes that can be sent to other participants. Furthermore, the app offers creators of events the possibility to add and remove participants and customize the event by giving it a title (and also being able to later edit it). The event creator can add and remove expenses from an event. Moreover, Splitty (the name of the application) also enables its users to switch between events so they can have a better overview of what they owe and are owed. Also, to have a better user experience, users can choose their preferred language (Dutch or English).
\\\\
To have a better user experience, we will make it easier for users to choose their preferred languages by having the flag of their country next to it (eg: for English, there will be the flag of the UK) and having a list of all the languages he/she can choose from. Also, the chosen language will persist throughout the user's sessions on the app. 
\\\\
 Users can assign dates to expenses for easy tracking and decide whether to split costs equally or within a subgroup. The app supports money transfers between users, facilitating debt settlement and event expenses. It provides an overview of all registered expenses, offering a clear picture of total expenditure. Users can filter this overview to display only their expenses or those involving them, making it easy to verify transactions. Each expense entry details the date, payer, amount, and involved participants, ensuring transparency. This app is a user-friendly tool for managing personal finances and promoting accountability and financial awareness.
 \\\\
Splitty allows users to manage expenses in their preferred currency. Users can configure their preferred currency, which is persisted in the configuration file, and specify a currency for each expense. The app automatically fetches exchange rates, converting expenses to the user’s preferred currency based on the date of the expense. The app supports at least USD, EUR, and CHF. The server communicates with the exchange rate service, providing a GET service for all connected clients to request conversion rates or conversions. The app caches requested exchange rates in a local file, reusing previously fetched rates for repeated requests with the same parameters. This app simplifies multi-currency expense tracking, providing users with a clear understanding of their expenses.
\\\\\
Our app calculates the minimal amount of transfers needed in a group, allowing users to mark debts as settled. Users can view a list of participants who still owe money, add bank accounts for bank transfers, and receive payment instructions with a summary and expandable details. This app simplifies debt settlement, making it easy for users to manage their finances.
Splitty also allows users to organize expenses by tags. Users can select a tag from a list when creating an expense, with three standard tags: food, entrance fees, and travel. Users can also add new tags to define their expense types. The app features a statistics page where users can explore the distribution of expenses for an event, see the total expense amount, and view a basic pie chart that provides an overview of the expenses per tag. Tags are always displayed with a colored background, and users can define or change the color of a tag, and rename or delete tags. The pie chart contains both absolute and relative values, offering a comprehensive view of how different types of expenses contribute to the overall cost. This app is a powerful tool for tracking and understanding expenses.
\\\\
Our app also allows users to configure their email credentials in the config file, enabling them to send notifications or reminders. A default email can be sent to test the validity of the credentials. The app allows users to send invitation emails to new participants, which automatically adds them to the event. These emails contain the server URL and an invitation code required to join. Users can add, change, or remove email addresses in each participant’s details. Payment invitations can be sent via a button click in the debt overview, simplifying the process of contacting participants. The app is usable even without or with invalid email credentials, with email-related features disabled when no credentials are available or the target participant has no known email address. The email configuration is stored only on the client and all sent emails include the user’s email address in CC.
\\\\
To help admins with maintenance, Splitty also features a password-protected management overview for server instance control. A random password is displayed in the server output for login. The app allows admins to view all server events, order them by title, creation date, or last activity, and delete events for database maintenance. It supports downloading a JSON dump of a selected event for backup purposes and importing an event from a JSON dump for backup restoration. This app simplifies server management, providing admins with a comprehensive overview and control of their server instances.
\\\\
After completing this project we will have learned to work with a team, communicate efficiently our thoughts and ideas and understand the people we are working with. Another learning objective will be the ability to present a product  (presenting it to the TA – project manager) and to communicate our progress and internal problems (related to team members – if we have any). We will also get better at solving different problems as a team and making use of each others’ strengths. Besides soft skills, our goal is to learn how to adapt to certain challenges that migth come along the way, how to learn new technologies in a timely manner. Furthermore, we will learn how to work with a database (store and retrieve information from it), how to organize a database, how to create a user interface and make it as user-friendly as possible. Moreover, we will also improve our planning skills and class design skills by making the code more robust and scalable.
