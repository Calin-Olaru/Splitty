\section{Dealing With Conflicts}
Conflicts should first of all be prevented at all times, as they do not have to be dealt with if they don’t arise in the first place. If however, a conflict arises at any point in time, a set-in-stone procedure shall be followed: First of all, we shall try to resolve the issue in a private matter, one-on-one preferably face to face, but otherwise via only communication channels. If, however, this shows to not be effective enough, then one should take the issue to the entire group, for example in the aforementioned Tuesday or Friday meetings, or via the online communication channel (Discord or WhatsApp). If this also shows to not resolve the conflict, then one should try to contact the TA, via so-called “On the record” communication channels, these being via Mattermost or a face-to-face conversation (preferably during the Tuesday meeting).
\\\\
When in the process of resolving a conflict between members of the team, members should in no single possible instance feel the need to resort to vulgar or insulting language, thus keeping every altercation between members of the team formal. This does, however, not mean that every interaction between members of the team has to be formal, use of vulgar language can be used in a non-threatening, non-aggressive way like, for example, speaking in hyperbole or using it to emphasize words.
